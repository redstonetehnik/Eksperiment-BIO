\documentclass[12pt, a4paper, oneside]{report}

\usepackage[utf8]{inputenc}
\usepackage[slovene]{babel}
\usepackage[margin=2.5cm, left=3cm]{geometry}
\usepackage{etoolbox}
\usepackage{fancyhdr}
\usepackage{titlesec}
\usepackage{array}
%\usepackage{apacite}
\usepackage{url}
\usepackage{biblatex-apa}
%\usepackage{tablefootnote}
%\usepackage{latexmk}

\DeclareLanguageMapping{slovene}{slovene-apa}
\bibliography{Eksperiment}

\makeatletter
\patchcmd{\chapter}{\if@openright\cleardoublepage\else\clearpage\fi}{}{}{}
\makeatother
%this is to remove pagebreak after chapter

\titleformat{\chapter}[display]
  {\normalfont\huge\bfseries}{}{0pt}{\Huge}
\titlespacing*{\chapter}
  {0pt}{10pt}{40pt}
%this is to remove Chapter #X that appeared before each chapter

\newcolumntype{H}{>{\lrbox0}c<{\endlrbox}@{}}
%the H colums will now be hidden

\newcommand{\comment}[2]{#2}

\begin{document}

\title{KAKO UMAZANE SO STVARI V RESNICI}
\author{Vladimir Smrkolj\\Lenart Bučar\\Gregor Gajič}
\date{marec, 2017}

\sffamily

\begin{titlepage}
\centering


{\Large{GIMNAZIJA BEŽIGRAD\\[2mm]LJUBLJANA}}

\vspace{3.5cm}

{\large Poročilo o izvedenem poskusu

\vspace{1cm}

BIOLOGIJA
}

\vspace{3.5cm}

\makeatletter
{\LARGE{\textbf{\@title}}}
\makeatother


\vfill


\parbox{4cm}{Mentor: Petra Starbek}
\hfill
\makeatletter
Kandidati: \parbox[t]{3cm}{\@author}
\makeatother

\vspace{3cm}

\makeatletter
Ljubljana, \@date
\makeatother

\end{titlepage}


\tableofcontents

\begin{abstract}
\begin{description}

\item[Teoretična izhodišča:]

\item[Metoda:]

\item[Rezultati:]

\item[Razprava:]

\item[Ključne besede:]

\end{description}
\end{abstract}


\pagestyle{fancy}
\lhead{Gimnazija Bežigrad}
\chead{}
\rhead{Poročilo poskusa}


\chapter{UVOD}

\section{\textsc{Namen Naloge}}

Namen te naloge je spoznati ali so stvari, za katere mislimo, da so umazane, res ``umazane'' in ali se med predmeti, ki jih uporabljamo vsak dan, skrivajo predmeti, ki izgledajo čisti, vendar so v resnici zelo ``umazani''.

\chapter{TEORETIČNI DEL -- UMAZANOST}

\section{\textsc{Bakterije}}

Bakterije so enocelični organizmi, ki jih po načinu prehranjevanja razlikujemo na avtotrofe in heterotrofe. \textbf{Avtotrofi} so organizmi, ki kot edini vir ogljika uporabljajo ogljikov dioksid (primer so cianobakterije), \textbf{heterotrofi} pa potrebujejo ogljik v organski obliki.

Bakterije najdemo skoraj povsod: v zemlji, vodi, zraku, tudi na rastlinah, živalih in človeku. Pri živalih in človeku žive bakterije na koži, v nosu, na zunanjih spolovilih in v prebavnem traktu, kjer so celo koristne, ker s svojimi encimi pomagajo pri prebavi, same pa gostitelja oskrbujejo z nekaterimi vitamini. Tako govorimo o normalni mikroflori, ki delno deluje tudi kot zaščita pred boleznimi, ker s škodljivimi mikrobi tekmuje za življenjski prostor in hrano. Če pridejo v organizem škodljive patogene bakterije, jih organizem skuša uničiti s svojimi obrambnimi mehanizmi. Kadar mu to ne uspe, se začno bakterije razmnoževati, nekatere izločajo tudi strupe; človek ali žival zboli in lahko tudi umre.

Tako nevarne kot nenevarne bakterije se lahko prenašajo na fizičnem prenosniku. Fizični prenosnik je lahko vsaka stvar, bakterije pa lahko z njega odstranimo s čiščenjem, dezinfekcijo in sterilizacijo:
\begin{itemize}
\item Čiščenje je postopek, pri katerem odstranimo vidno umazanijo, organske ostanke in del bakterij.
\item Dezinfekcija ali razkuževanje je postopek, pri katerem uničimo vse vegetativne oblike bakterij. Razkuževanje izvajamo na tri načine, in sicer mehansko (drgnjenje, brisanje), kemično (razkužila) in s fizikalnimi postopki (UV sevanje, segrevanje).
\item Sterilizacija je postopek, s katerim uničimo vse vegetativne oblike in spore bakterij ter vse viruse.
\end{itemize}

V ameriški raziskavi so ugotovili, da je na vsakem kvadratnem centimetru mobilnega telefona skoraj 4000 mikroorganizmov. Odkrili so tudi, da je na 94,5\,\% mobilnih telefonov vsaj ena vrsta bakterij, mnogo teh pa je odpornih na več antibiotikov. S primerjanjem brisa rok uporabnikov in njihovih mobilnih telefonov so odkrili, da se večina bakterij prenese z rok na mobilne telefone in obratno. V drugi ameriški raziskavi so zapisali, da lahko mobilni telefoni zdravstvenih delavcev služijo kot prenašalci bakterij iz enega bolnika na drugega. Predlagali so, da se prepove uporabo mobilnih telefonov v kliničnih prostorih, kjer je potrebna popolna asepsa (operacijska soba) oziroma je nevarnost prenašanja mikroorganizmov večja.

\section{\textsc{Gojenje mikroorganizmov}}

Gojenje mikroorganizmov je osnova za njihovo proučevanje. Rast mikroorganizmov pomeni rast celic in povečevanje števila celic. Za rast je treba zagotoviti hranila, vir energije in ustrezne fizikalno-kemijske razmere. Gojišča in razmere, v katerih kulturo gojimo, morajo čim bolj posnemati naravno okolje.

\section{\textsc{Gojišča}}

Gojišča so substrati za gojenje mikroorganizmov v laboratorijskih razmerah. Gojišča, ki jih uporabljamo v mikrobiologiji so številna in raznolika, saj jih prilagajamo potrebam mikroorganizmov in načinu uporabe (izolacija, vzdrževanje, namnoževanje in karakterizacija kulture).

Gojišča so vodne raztopine snovi, ki jih določen mikroorganizem potrebuje za rast. Glede na sestavo jih delimo na kompleksna in definirana. \textbf{Kompleksna gojišča} so tista, pri katerih ne poznamo njihove natančne kemijske sestave. \textbf{Definirana gojišča} vsebujejo točno določene koncentracije čistih kemikalij.

Vsako gojišče lahko pripravimo kot tekoče, trdno ali poltrdno glede na količino dodanih snovi, ki povzročajo strjevanje gojišča. Običajno je to agar, redkeje želatina ali silikagel.

Glede na vrsto in namen gojišča, jih delimo na:
\begin{description}
\item[Osnovna gojišča] so bogata kompleksna gojišča, na katerih raste večina heterotrofnih bakterij.
\item[Obogatena gojišča] dobimo, če osnovnim gojiščem primešamo različne dodatke, ki so vir aminokislin, vitaminov, purinov, pirimidinov in drugih neznanih rastnih dejavnikov. Uporabljamo jih za izolacijo in gojenje prehransko zahtevnejših bakterij. Najbolj splošno obogateno gojišče je krvni agar, ki ga pripravimo tako, da v osnovno gojišče dodamo 5\,\% krvi. Če kri dodamo v vroče gojišče, dobimo čokoladni agar. Gojišča lahko obogatimo tudi s krvnim serumom, jajci, posnetim mlekom.
\item[Diferencialna gojišča] omogočajo rast večini bakterij, vendar se na njih kolonije različnih bakterijskih vrst razlikujejo med seboj.
\item[Selektivna gojišča] so sestavljena tako, da omogočajo rast samo določeni skupini bakterij (negativna selekcija). To dosežemo z dodajanjem snovi, ki zavirajo rast večine drugih mikroorganizmov (antibiotiki, barvila, NaCl).
\item[Bogatitvena gojišča] enako kot selektivna omogočajo rast samo določenim mikroorganizmom, ker vsebujejo specifične vire ogljika ali drugih potrebnih elementov, ki omogočajo rast samo zaželenim mikrobom (pozitivna selekcija).
\item[Produkcijska gojišča] se uporabljajo v industriji.
\item[Gojišča za ugotavljanje lastnosti bakterij] so sestavljena tako, da lahko ugotavljamo dejavnost različnih encimov, sposobnost bakterije, da razgrajuje različne substrate in podobno.
\end{description}

\subsection{\textsc{Agar}}

\textbf{Agar} je polisaharid iz galaktoze in galakturonske kisline, ki ga pridobivajo iz določenih vrst rdečih morskih alg. V večino poltrdih gojišč je dodano 1,5\,\% agarja. V mrzli vodi nabrekne, pri 100oC pa se popolnoma raztopi in s tem preide v koloidno stanje sol. Pri ohlajanju pod temperaturo 40oC preide ponovno v stanje gel. Želatina, ki se včasih uporablja namesto agarja, preide v stanje sol že pri 37oC, to pa je za mnoge bakterije najugodnejša temperatura za rast. Le redke morske bakterije hidrolizirajo agar. Pri gojenju teh uporabljajo silikagel gojišča. Prvi agar so je pojavil leta 1658 na Japonskem, odkril ga je Mino Tarōzaemon, ki je opazil, da se je juha, narejena iz morskih alg, strdila v gel, ko se je ponoči temperatura juhe znižala. Agar se je v biologiji prvič uporabil leta 1882. Uporabil ga je nemški mikrobiolog Walther Hesse v laboratoriju Roberta Kocha.

\section{\textsc{Mikrobna Kultura}}

Mikrobna kultura so mikroorganizmi, ki porastejo v gojišču ali na njem. Mikrobna kultura je lahko mešana ali čista. V \textbf{čisti kulturi} je samo ena vrsta mikroorganizmov in zato so lastnosti kulture lastnosti te vrste mikroorganizma. V \textbf{mešani kulturi} je več kot ena vrsta mikroorganizmov.

\section{\textsc{Nacepljanje Mikroorganizmov Na/V Gojišča}}

Mikroorganizme lahko nacepljamo na trdna ali v tekoča gojišča. Za nacepljanje ali inokulacijo uporabljamo cepilno zanko (eza), lahko pa tudi vatenko ali pipeto. Če nacepljamo mikroorganizme na trdna gojišča, morajo biti kolonije po inkubaciji dovolj narazen, da lahko zrastejo do za vrsto značilne velikosti in da razvijejo druge značilnosti. Tak način cepljenja imenujemo redko cepljenje ali \textbf{cepljenje do posameznih kolonij}.

\section{\textsc{Redčenje Kulture Do Posameznih Kolonij}}

Posamezne kolonije dobimo, če kulturo ali preiskovani material nacepimo tako, da ga ``razredčujemo'' na površini trdnega gojišča v petrijevki. Kulturo nanesemo na majhno površino ob robu gojišča. Vsako nadaljnjo serijo potez začnemo s sterilno cepilno zanko.

\section{Umazanost}

\chapter{RAZISKOVALNI DEL}

\section{Namen in cilji raziskovanja}
Namen te naloge je spoznati ali so stvari, za katere mislimo, da so »umazane«, torej tudi onesnažene z velikim številom mikroorganizmov, res »umazane« in ali se med predmeti, ki jih uporabljamo vsak dan, skrivajo predmeti, ki izgledajo čisti, vendar so v resnici zelo »umazani«.

\section{Raziskovalne hipoteze, Raziskovalna vprašanja}

Površina mobilnega telefona je najmanj čista od površin osmih predmetov, s katerih bomo vzeli brise (površina straniščne deske, mobilni telefon, tipke na računalniški tipkovnici, tipke na žepnem računalu, kljuka na vratih, bankovec, kovanec, tipkovnica na bankomatu).

\section{Raziskovalna metodologija}

\subsection{Metode in tehnike zbiranja podatkov}

Za uspešno izvedbo poskusa potrebujemo sterilno opremo in sterilni agar. Agar bomo sterilizirali po naslednjem postopku: Najprej bomo v 500\,ml erlenmajerico vložili 250\,g agarja. Okoli vratu erlenmajerice bomo z aluminijasto folijo naredili zaviti vrat, ki bo gledal navzdol, tako da bo preprečil vstop kondenzirane vode v erlenmajerico. V lonec na zvišan pritisk bomo nalili okoli 2-3\,cm vode. Na podstavek v posodi na zvišan pritisk bomo postavili erlenmajerico, tako da se le-ta ne bo dotikala vode. V loncu na zvišan pritisk bomo sterilizirali agar vsaj 45 minut pri 121$^\circ$\,C oziroma pri 1,05\,bara. Po 45 minutah bomo lonec odstranili z grelca in pokrili pokrov s čisto in mokro krpo, zato da bomo preprečili vstop delcev med ohlajanjem. Ko se bo pritisk znotraj lonca izenačil z zunanjim pritiskom, bomo odprli pokrov in prelili agar v sterilne petrijevke. V vsako petrijevko bomo nalili 25-30\,ml tekočega agarja. Petrijevke bomo pokrili in počakali, da se sloj agarja strdi, nato so petrijevke pripravljene na uporabo.

Potem bomo sterilne vatirane palčke namočili v sterilno fiziološko raztopino in s tako pripravljenimi vatiranimi palčkami odvzeli bris izbrane površine. Ko bomo zbrali brise vseh 8 površin, bomo razmazali kužnine po delu hranilnega agarja na predpripravljeni petrijevki. S cepilno zanko bomo naredili nekaj pravokotnih potez čez prvi premaz, nato bomo nadaljevali v isti smeri cepljenja, ne da bi se dotikali prvega premaza. Petrijevke bomo za tem označili in jih inkubirali dva dni pri temperaturi 37$^\circ$\,C. Po dveh dneh inkubiranja bomo prešteli kolonije bakterij v vsaki petrijevki in si zapisali rezultate ter opažanja. S pomočjo rezultatov bomo primerjali čistočo površin oz. predmetov, s katerih smo odvzeli bris.

\subsection{Opis istrumentarija}

Za uspešno izvedbo tega poskusa potrebujemo:
\begin{itemize}
  \item Sterilne petrijevke z agarjem za gojenje bakterij
  \item 250g agarja za gojenje bakterij
  \item Prenosne sterilne vatirane palčke za odvzem brisa
  \item Površine, s katerih bomo odvzeli brise:
  \begin{itemize}
    \item Površina straniščne deske
    \item Mobilni telefon
    \item Tipke na računalniški tipkovnici
    \item Tipke na žepnem računalu
    \item Kljuka na vratih
    \item Bankovec
    \item Kovanec
    \item Tipkovnica na bankomatu
  \end{itemize}
  \item Sterilno fiziološko raztopino
  \item Inkubator oz. prostor z nadzorovano temperaturo zraka
  \item Cepilne zanke (eze)
  \item Laboratorijski gorilnik
  \item Razkužilo
  \item Zaščitne rokavice

\end{itemize}

\subsection{Opis vzorca}

\subsection{Opis obdelave podatkov}

\section{REZULTATI}

V sledečih tabelah so prikazani opisi kolonij za vsako gojišče posebej.

\begin{table}%[p]
  \textbf{\caption{Št. kolonij v odvisnosti od datuma}}
  \begin{tabular}{||l||r|r|r|r|r|rHHHHHHHH|}
    \hline \hline
    Datum & 9.\,2.\,2017 & 10.\,2.\,2017 & 10.\,2.\,2017 & 11.\,2.\,2017 & 12.\,2.\,2017 & 13.\,2.\,2017 & 14.\,2.\,2017 & 15.\,2.\,2017 & 16.\,2.\,2017 & 17.\,2.\,2017 & 18.\,2.\,2017 & 19.\,2.\,2017 & 20.\,2.\,2017 & 21.\,2.\,2017 \\ \hline
    Ura & 17.51	& 6.48	& 18.10	& 17.34	& 17.24	& 17.57	& 20.14	& 17.23	& 18.10	& 18.22	& 18.02	& 16.56	& 21.29	& 19.05 \\ \hline \hline
    SD & 18 & 19 & 17 & 23 & 23 & 24 & 25 & 25 & 28 & 38 & 39 & 38 & 37 & 37 \\ \hline
    MT & 32 & 35 & 35 & 35 & 34 & 39 & 43 & 42 & 41 & 45 & 48 & 47 & 47 & 48 \\ \hline
    RT & 4 & 4 & 4 & 4 & 5 & 5 & 5 & 6 & 6 & 7 & 7 & 7 & 9 & 8 \\ \hline
    ŽR & 6 & 6 & 6 & 6 & 8 & 8 & 9 & 9 & 10 & 10 & 11 & 11 & 10 & 10 \\ \hline
    KV & 2 & 2 & 2 & 2 & 2 & 2 & 2 & 2 & 2 & 2 & 2 & 2 & 2 & 2 \\ \hline
    BA & 1 & 1 & 2 & 2 & 2 & 3 & 3 & 3 & 3 & 3 & 3 & 3 & 3 & 3 \\ \hline
    KO & 0 & 0 & 0 & 0 & 1 & 1 & 1 & 1 & 1 & 1 & 1 & 1 & 2 & 2 \\ \hline
    KO & 0 & 0 & 0 & 0 & 1 & 1 & 1 & 1 & 1 & 1 & 1 & 1 & 2 & 2 \\ \hline \hline
  \end{tabular}
  \vspace{0.3cm}
  \begin{tabular}{|HHHHHHHr|r|r|r|r|rHH|}
    \hline \hline
    Datum & 9.\,2.\,2017 & 10.\,2.\,2017 & 10.\,2.\,2017 & 11.\,2.\,2017 & 12.\,2.\,2017 & 13.\,2.\,2017 & 14.\,2.\,2017 & 15.\,2.\,2017 & 16.\,2.\,2017 & 17.\,2.\,2017 & 18.\,2.\,2017 & 19.\,2.\,2017 & 20.\,2.\,2017 & 21.\,2.\,2017 \\ \hline
    Ura & 17.51	& 6.48	& 18.10	& 17.34	& 17.24	& 17.57	& 20.14	& 17.23	& 18.10	& 18.22	& 18.02	& 16.56	& 21.29	& 19.05 \\ \hline \hline
    SD & 18 & 19 & 17 & 23 & 23 & 24 & 25 & 25 & 28 & 38 & 39 & 38 & 37 & 37 \\ \hline
    MT & 32 & 35 & 35 & 35 & 34 & 39 & 43 & 42 & 41 & 45 & 48 & 47 & 47 & 48 \\ \hline
    RT & 4 & 4 & 4 & 4 & 5 & 5 & 5 & 6 & 6 & 7 & 7 & 7 & 9 & 8 \\ \hline
    ŽR & 6 & 6 & 6 & 6 & 8 & 8 & 9 & 9 & 10 & 10 & 11 & 11 & 10 & 10 \\ \hline
    KV & 2 & 2 & 2 & 2 & 2 & 2 & 2 & 2 & 2 & 2 & 2 & 2 & 2 & 2 \\ \hline
    BA & 1 & 1 & 2 & 2 & 2 & 3 & 3 & 3 & 3 & 3 & 3 & 3 & 3 & 3 \\ \hline
    KO & 0 & 0 & 0 & 0 & 1 & 1 & 1 & 1 & 1 & 1 & 1 & 1 & 2 & 2 \\ \hline
    KO & 0 & 0 & 0 & 0 & 1 & 1 & 1 & 1 & 1 & 1 & 1 & 1 & 2 & 2 \\ \hline \hline
  \end{tabular}
  \vspace{0.3cm}
  \begin{tabular}{|HHHHHHHHHHHHHr|r||}
    \hline \hline
    Datum & 9.\,2.\,2017 & 10.\,2.\,2017 & 10.\,2.\,2017 & 11.\,2.\,2017 & 12.\,2.\,2017 & 13.\,2.\,2017 & 14.\,2.\,2017 & 15.\,2.\,2017 & 16.\,2.\,2017 & 17.\,2.\,2017 & 18.\,2.\,2017 & 19.\,2.\,2017 & 20.\,2.\,2017 & 21.\,2.\,2017 \\ \hline
    Ura & 17.51	& 6.48	& 18.10	& 17.34	& 17.24	& 17.57	& 20.14	& 17.23	& 18.10	& 18.22	& 18.02	& 16.56	& 21.29	& 19.05 \\ \hline \hline
    SD & 18 & 19 & 17 & 23 & 23 & 24 & 25 & 25 & 28 & 38 & 39 & 38 & 37 & 37 \\ \hline
    MT & 32 & 35 & 35 & 35 & 34 & 39 & 43 & 42 & 41 & 45 & 48 & 47 & 47 & 48 \\ \hline
    RT & 4 & 4 & 4 & 4 & 5 & 5 & 5 & 6 & 6 & 7 & 7 & 7 & 9 & 8 \\ \hline
    ŽR & 6 & 6 & 6 & 6 & 8 & 8 & 9 & 9 & 10 & 10 & 11 & 11 & 10 & 10 \\ \hline
    KV & 2 & 2 & 2 & 2 & 2 & 2 & 2 & 2 & 2 & 2 & 2 & 2 & 2 & 2 \\ \hline
    BA & 1 & 1 & 2 & 2 & 2 & 3 & 3 & 3 & 3 & 3 & 3 & 3 & 3 & 3 \\ \hline
    KO & 0 & 0 & 0 & 0 & 1 & 1 & 1 & 1 & 1 & 1 & 1 & 1 & 2 & 2 \\ \hline
    KO & 0 & 0 & 0 & 0 & 1 & 1 & 1 & 1 & 1 & 1 & 1 & 1 & 2 & 2 \\ \hline \hline
  \end{tabular}
\end{table}


\begin{table}
  \textbf{\caption{Kovanec}}
  \begin{tabular}{||l|c||}
    \hline \hline
    Velikost & \textless\,1\,mm \\ \hline
    Velikost (opisno) & točkasta \\ \hline
    Barva & prozorna \\ \hline
    Površina & sijoča \\ \hline
    Tekstura & / \\ \hline
    Št. kolonij & 2 \\ \hline
    Oblika & okrogle \\ \hline
    Rob & / \\ \hline
    Vzdignjenost & / \\ \hline
    Optične lastnosti & prozorna \\ \hline \hline
  \end{tabular}
\end{table}

\begin{table}
  \textbf{\caption{Straniščna Deska}}
  \begin{tabular}{||l|c||}
    \hline \hline
    Velikost & 5\,mm \\ \hline
    Velikost (opisno) & srednja \\ \hline
    Barva & mlečna \\ \hline
    Površina & mat \\ \hline
    Tekstura & gladka in suha \\ \hline
    Št. kolonij & 12 \\ \hline
    Oblika & okrogle \\ \hline
    Rob & gladek \\ \hline
    Vzdignjenost & ploske \\ \hline
    Optične lastnosti & prosojna \\ \hline \hline
  \end{tabular}
\end{table}

\begin{table}
  \textbf{\caption{Žepno Računalo}}
  \begin{tabular}{||l|c||}
    \hline \hline
    Velikost & 5\,mm \\ \hline
    Velikost (opisno) & majhna \\ \hline
    Barva & mlečna/rumenkasta \\ \hline
    Površina & sijoča \\ \hline
    Tekstura & gladka/vlažna \\ \hline
    Št. kolonij & 5 \\ \hline
    Oblika & okrogle \\ \hline
    Rob & gladek \\ \hline
    Vzdignjenost & umbilicirane \\ \hline
    Optične lastnosti & motna/prosojna \\ \hline \hline
  \end{tabular}
\end{table}

\begin{table}
  \textbf{\caption{Tipke na bankomatu}}
  \begin{tabular}{||l|c||}
    \hline \hline
    Velikost & 2\,mm \\ \hline
    Velikost (opisno) & majhna \\ \hline
    Barva & prozorna \\ \hline
    Površina & sijoča \\ \hline
    Tekstura & hrapava \\ \hline
    Št. kolonij & 1 \\ \hline
    Oblika & nepravilne \\ \hline
    Rob & valovit \\ \hline
    Vzdignjenost & umbilicirane \\ \hline
    Optične lastnosti & prozorna \\ \hline \hline
  \end{tabular}
\end{table}


\begin{table}
  \textbf{\caption{Računalniška tipkovnica}}
  \begin{tabular}{||l|c||}
    \hline \hline
    Velikost & 8\,mm \\ \hline
    Velikost (opisno) & velika \\ \hline
    Barva & mlečna \\ \hline
    Površina & sijoča \\ \hline
    Tekstura & gladka \\ \hline
    Št. kolonij & 2 \\ \hline
    Oblika & okrogle \\ \hline
    Rob & naguban \\ \hline
    Vzdignjenost & ploske \\ \hline
    Optične lastnosti & motna/prosojna \\ \hline \hline
  \end{tabular}
\end{table}

\begin{table}
  \textbf{\caption{Bankovec}}
  \begin{tabular}{||l|c||}
    \hline \hline
    Velikost & 10\,mm \\ \hline
    Velikost (opisno) & velika \\ \hline
    Barva & mlečna \\ \hline
    Površina & mat \\ \hline
    Tekstura & gladka \\ \hline
    Št. kolonij & 1 \\ \hline
    Oblika & okrogle \\ \hline
    Rob & gladek \\ \hline
    Vzdignjenost & ploske \\ \hline
    Optične lastnosti & prosojna \\ \hline \hline
  \end{tabular}
\end{table}

\begin{table}
  \textbf{\caption{Kljuka na vratih}}
  \begin{tabular}{||l|c||}
    \hline \hline
    Velikost & 8\,mm \\ \hline
    Velikost (opisno) & velika \\ \hline
    Barva & mlečna \\ \hline
    Površina & mat \\ \hline
    Tekstura & suha \\ \hline
    Št. kolonij & 1 \\ \hline
    Oblika & okrogle \\ \hline
    Rob & naguban \\ \hline
    Vzdignjenost & ploske \\ \hline
    Optične lastnosti & motna \\ \hline \hline
  \end{tabular}
\end{table}

\begin{table}
  \textbf{\caption{Mobilni telefon}}
  \begin{tabular}{||l|c||}
    \hline \hline
    Velikost & 2\,mm \\ \hline
    Velikost (opisno) & majhna \\ \hline
    Barva & kremasto-oranžna \\ \hline
    Površina & mat \\ \hline
    Tekstura & hrapava \\ \hline
    Št. kolonij & 1 \\ \hline
    Oblika & okrogle \\ \hline
    Rob & naguban \\ \hline
    Vzdignjenost & dvignjene \\ \hline
    Optične lastnosti & prosojna \\ \hline \hline
  \end{tabular}
\end{table}

\chapter{DISKUSIJA}

\chapter{ZAKLJUČEK}

\chapter{PRILOGE}

\chapter{VIRI, LITERATURA}
%\begingroup
%\renewcommand{\section}[2]{}%
%\renewcommand{\chapter}[2]{}% for other classes
%\bibliographystyle{apacite}
\printbibliography
%\nocite{*}




















































\end{document}
